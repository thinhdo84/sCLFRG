
\documentclass[11pt]{article}
\usepackage{amsmath,amssymb,geometry}
\geometry{margin=1in}
\title{Theory of Model Predictive Control on $\mathrm{SO}(3)$ via Rotation Vectors}
\author{}
\date{}

\begin{document}

\maketitle

\section*{1. Background: $\mathrm{SO}(3)$ and Rotation Vectors}

\subsection*{1.1 Definition}
The special orthogonal group in 3D is defined as:
\[
\mathrm{SO}(3) = \left\{ R \in \mathbb{R}^{3 \times 3} \;\middle|\; R^\top R = I,\ \det(R) = 1 \right\}
\]
This group represents 3D rotation matrices. It forms a Lie group with tangent space at the identity given by the Lie algebra:
\[
\mathfrak{so}(3) = \left\{ \hat{\omega} \in \mathbb{R}^{3 \times 3} \mid \hat{\omega}^\top = -\hat{\omega} \right\}
\]

\subsection*{1.2 Hat and Vee Operators}
Given a vector \( \omega = [\omega_1, \omega_2, \omega_3]^\top \in \mathbb{R}^3 \), the \emph{hat map} is:
\[
\hat{\omega} =
\begin{bmatrix}
0 & -\omega_3 & \omega_2 \\
\omega_3 & 0 & -\omega_1 \\
-\omega_2 & \omega_1 & 0
\end{bmatrix}
\]
This maps \( \mathbb{R}^3 \to \mathfrak{so}(3) \). The \emph{vee map} \( (\cdot)^\vee \) is its inverse:
\[
(\hat{\omega})^\vee = \omega
\]

\subsection*{1.3 Exponential Map and Rotation Vectors}
The exponential map from \( \mathfrak{so}(3) \to \mathrm{SO}(3) \) gives:
\[
\exp(\hat{\omega}) = I + \frac{\sin \theta}{\theta} \hat{\omega}
+ \frac{1 - \cos \theta}{\theta^2} \hat{\omega}^2,\quad \theta = \|\omega\|
\]
This represents a rotation by \( \theta \) radians around the axis \( \omega / \|\omega\| \). Thus, any rotation can be represented as:
\[
R = \exp(\hat{\phi})
\]
where \( \phi \in \mathbb{R}^3 \) is the \emph{rotation vector}.

\section*{2. Rigid Body Dynamics}

\subsection*{2.1 Continuous-Time Model}
For a rigid body in body frame coordinates:
\[
\dot{R} = R \hat{\omega}, \quad J \dot{\omega} = \tau - \omega \times J\omega
\]
where:
\begin{itemize}
    \item \( R \in \mathrm{SO}(3) \) is the orientation matrix
    \item \( \omega \in \mathbb{R}^3 \) is the angular velocity
    \item \( J \in \mathbb{R}^{3 \times 3} \) is the inertia matrix
    \item \( \tau \in \mathbb{R}^3 \) is the control torque
\end{itemize}

\section*{3. Model Predictive Control using Rotation Vectors}

\subsection*{3.1 Rotation Error Representation}
To track a desired rotation \( R_d \), define the rotation error:
\[
R_e = R_d^\top R = \exp(\hat{\phi})
\Rightarrow \phi = \log(R_d^\top R)^\vee
\]
This defines the minimal 3D rotation vector \( \phi \in \mathbb{R}^3 \) that brings \( R \) to \( R_d \).

\subsection*{3.2 Rotation Vector Dynamics}
We locally approximate:
\[
\dot{\phi} = \omega, \quad \dot{\omega} = J^{-1}(\tau - \omega \times J\omega)
\]
Discretized with timestep \( h \), using Euler method:
\[
\phi_{k+1} = \phi_k + h \cdot \omega_k \\
\omega_{k+1} = \omega_k + h \cdot J^{-1}(\tau_k - \omega_k \times J\omega_k)
\]

\subsection*{3.3 MPC Problem Formulation}
\begin{align*}
\min_{\{\tau_k\}_{k=0}^{N-1}} \quad & \sum_{k=0}^{N-1} \left( \|\phi_k\|^2 + \lambda \|\tau_k\|^2 \right) \\
\text{s.t.} \quad
& \phi_{k+1} = \phi_k + h \omega_k \\
& \omega_{k+1} = \omega_k + h J^{-1}(\tau_k - \omega_k \times J\omega_k) \\
& \tau_{\min} \leq \tau_k \leq \tau_{\max} \\
& \phi_0 = \phi_{\text{init}},\ \omega_0 = \omega_{\text{init}}
\end{align*}

This problem is convex in cost, but nonlinear in dynamics. It can be solved using CasADi + IPOPT in MATLAB.

\section*{4. Practical Notes}
\begin{itemize}
    \item Using rotation vectors avoids manifold constraints (like \( R^\top R = I \))
    \item Rotation errors remain small under MPC horizon, making the linearized model valid
    \item Control torques must be bounded for physical realism
    \item No terminal cost or constraint is needed for local stabilization
\end{itemize}

\end{document}
